\documentclass[12pt,a4paper]{article}

\usepackage[utf8]{inputenc}
\usepackage[T1]{fontenc}
\usepackage{amsmath,amssymb,amsthm}
\usepackage{mathtools}
\usepackage{geometry}
\usepackage{hyperref}
\usepackage{enumitem}
\usepackage{xcolor}
\usepackage{fancyhdr}
\usepackage{titlesec}
\usepackage{abstract}

\geometry{left=2.5cm,right=2.5cm,top=2.5cm,bottom=2.5cm}
\hypersetup{colorlinks=true,linkcolor=blue!70!black,citecolor=green!50!black,urlcolor=blue!70!black}

\newtheorem{theorem}{Theorem}[section]
\newtheorem{proposition}[theorem]{Proposition}
\theoremstyle{definition}
\newtheorem{definition}[theorem]{Definition}

\newcommand{\R}{\mathbb{R}}
\newcommand{\C}{\mathbb{C}}
\newcommand{\HH}{\mathbb{H}}
\newcommand{\OO}{\mathbb{O}}
\newcommand{\flatmod}{\flat}
\newcommand{\nextmod}{\bigcirc}

\titleformat{\section}{\large\bfseries}{\thesection.}{0.5em}{}
\titleformat{\subsection}{\normalsize\bfseries}{\thesubsection}{0.5em}{}

\pagestyle{fancy}
\fancyhf{}
\fancyhead[L]{\small Structural Emergence of 4D Lorentzian Geometry}
\fancyhead[R]{\small H.S.~Lande}
\fancyfoot[C]{\thepage}
\renewcommand{\headrulewidth}{0.4pt}

\begin{document}

\begin{center}
    {\LARGE\bfseries Structural Emergence of 4D Lorentzian Geometry\\[0.3em] from the Principle of Efficient Novelty}

    \vspace{1.5em}

    {\large Halvor S.~Lande}\\[0.5em]
    {\normalsize \href{mailto:hsl@awc.no}{hsl@awc.no}}

    \vspace{1em}

    {\normalsize March 1, 2026}
\end{center}

\vspace{1.5em}

\begin{abstract}
\noindent
We derive quantum kinematics, four-dimensionality, Lorentzian signature, and Einstein dynamics from the Principle of Efficient Novelty (PEN) using the efficiency ratio $\rho=\nu/\kappa$. The argument is presented in a constructive mechanization-oriented form with explicit artifact support and consistent realization indexing across the companion PEN materials. The resulting claim is: (i) linear/quantum kinematics is selected because Cartesian cloning collapses higher homotopical novelty; (ii) $d=4$ is selected by a native-associativity cost ceiling plus a 4D Hodge self-duality gain; (iii) Lorentzian signature is required to prevent univalent modal collapse of Flow and Cohesion; and (iv) Einstein--Hilbert dynamics is selected as the minimal abstract-syntax scalar curvature action.
\end{abstract}

\vspace{1em}
\hrule
\vspace{2em}

\section{Introduction}

\subsection{Architecture as a Selection Problem}

Physics is formulated inside a very specific container: a 4D manifold, Lorentzian signature $(-+++)$, and quantum state evolution. Standard presentations treat these as independent axioms. PEN instead treats them as outputs of one optimization criterion:
\begin{equation}
\rho = \frac{\nu\;\text{(novel derivation schemas)}}{\kappa\;\text{(specification and coherence cost)}}.
\end{equation}

The key claim is not ``anything is possible and we happened to land here,'' but ``this architecture is selected because nearby alternatives have lower $\rho$.''

\subsection{Index Alignment with the Companion Paper}

We follow realization numbering from \emph{Deriving the Standard Model from PEN}: Frame/Metric is \textbf{R13}, Hilbert functional is \textbf{R14}, and Dynamical Cohesive Topos (DCT) is \textbf{R15}. All references below use that indexing consistently.

\section{Quantum Kinematics from Efficient Novelty}

\subsection{No-Cloning Recast as a Type-Theoretic Constraint}

A previous formulation conflated copying data with generating new derivation schemas. Here novelty remains strictly logical: $\nu$ counts genuinely new inference/program schemas. In that sense, unrestricted Cartesian copying is not ``free novelty''; it is a degeneracy pressure.

In homotopy type theory, imposing strict global cloning coherence drives types toward h-sets (0-groupoids), suppressing higher path structure \cite{hottbook}. That collapse shrinks higher-geometric novelty channels and reduces long-run $\nu$. The operational no-cloning pressure in quantum theory is classically captured by standard no-cloning results \cite{wootters-zurek,dieks}, while resource-sensitive proof theory is rooted in linear logic \cite{girard-linear-logic}.

\subsection{Linearization and Stable Homotopy Gain}

Moving from Cartesian product structure toward linear/smash-like composition increases available generalized cohomological constructions, i.e., more nontrivial derivation families at bounded coherence overhead. This is the PEN reason for quantum/linear kinematics: it preserves resource sensitivity while retaining high homotopical generativity.

\begin{center}
\fbox{\parbox{0.92\textwidth}{
\textbf{Section conclusion.} Quantum kinematics is selected because it prevents cloning-induced homotopy collapse and supports a larger stable family of derivation schemas per unit specification cost.
}}
\end{center}

\section{Dimensionality: Why Four}

\subsection{R13 Native Associativity Ceiling (Corrected Frame-Bundle Argument)}

Pure differential geometry does \emph{not} cap dimension by division algebras: $GL(n,\R)$ exists for all $n$ and is associative. The corrected PEN statement is about \emph{native primitive cost}, not mathematical existence.

If the engine must synthesize full matrix infrastructure and coherence from scratch in high $n$, $\kappa$ grows rapidly. PEN therefore prefers maximal reuse of already-realized associative rotational primitives (from the Hopf stage): $\R,\C,\HH$ are strictly associative; $\OO$ is not. Using octonionic-like multiplication as a native patching primitive breaks strict cocycle composition unless one adds explicit higher coherence towers (effectively $A_\infty$ bookkeeping), which is costly.

Hence the native-associative ceiling is reached at quaternionic control, geometrically tied to
\begin{equation}
S^3 \hookrightarrow S^7 \to S^4,
\end{equation}
with $S^4$ as the minimal high-yield base dimension supported by reusable associative primitives.

\subsection{4D Hodge Endomorphism Bonus (Replacing Exotic-\texorpdfstring{$\R^4$}{R4} Overclaim)}

We remove the claim of a full formalized enumeration of exotic smooth structures. The tractable and sufficient 4D gain is algebraic:
\begin{equation}
\star : \Omega^2(M^d) \to \Omega^{d-2}(M^d).
\end{equation}
Only at $d=4$ does $\star$ close on 2-forms, giving a native endomorphism \cite{atiyah-hitchin-singer,donaldson-kronheimer}
\begin{equation}
\star : \Omega^2 \to \Omega^2,
\end{equation}
and immediate splitting into self-dual/anti-self-dual sectors. This doubles gauge-curvature interaction channels without adding new primitive constructors, i.e., substantial $\nu$ gain at near-constant $\kappa$.

\begin{center}
\fbox{\parbox{0.92\textwidth}{
\textbf{Section conclusion.} Dimension four is selected by the intersection of a native-associativity cost ceiling and a unique Hodge self-duality novelty bonus.
}}
\end{center}

\section{Signature: Lorentzian as Modal Separation Geometry}

\subsection{R15 Modal Structure and Univalence}

R15 (DCT) carries distinct modalities: Cohesion ($\flatmod$) for extended shape and Flow ($\nextmod$) for temporal update. A physical metric must realize this distinction rather than quotienting it away.

By univalence, sufficiently coherent geometric paths induce type equivalences \cite{hottbook}. In Euclidean signature $(++++)$, $SO(4)$ isotropy enables continuous rotation between any unit directions. If one direction is interpreted as Flow and another as Cohesion, isotropy supplies a path that univalence promotes toward equivalence, yielding modal collapse $\nextmod \simeq \flatmod$.

\subsection{Light Cone as a Type-Theoretic Barrier}

Lorentzian signature $(-+++)$ partitions tangent vectors into timelike/spacelike/null classes. Under $SO(1,3)$, timelike and spacelike regions are disconnected by the null cone; no continuous path crosses the barrier while preserving causal class \cite{hawking-ellis,wald-gr}. Therefore the specific univalent path needed to identify $\nextmod$ with $\flatmod$ is blocked.

\begin{center}
\fbox{\parbox{0.92\textwidth}{
\textbf{Section conclusion.} Lorentzian anisotropy is selected because it is the minimal geometry that protects modal separation in R15.
}}
\end{center}

\section{Dynamics: Einstein--Hilbert as AST-Minimal Curvature Law}

We replace an over-strong ``fully formalized Lovelock uniqueness'' claim with a computable minimality statement. Among local scalar curvature actions that yield second-order divergence-compatible equations in 4D, the Einstein--Hilbert term
\begin{equation}
S_{\mathrm{EH}} = \int (R - 2\Lambda)\,\sqrt{-g}\,d^4x
\end{equation}
has minimal abstract syntax depth/size. Terms like $R^2$ or $R_{\mu\nu}R^{\mu\nu}$ require strictly larger constructor trees, hence larger $\kappa$ unless forced by additional constraints; this is compatible with the classical Lovelock classification context in four dimensions \cite{lovelock-1971}.

\section{Predictions}

\begin{itemize}[leftmargin=2em]
    \item \textbf{No accessible extra dimensions:} higher-dimensional proposals pay native coherence overhead without compensating novelty gain in this framework.
    \item \textbf{No Lorentz-violation regime:} breaking Lorentzian structure reopens modal-collapse channels.
    \item \textbf{Minimal gravity sector first:} leading UV-consistent corrections, if any, should appear as controlled high-$\kappa$ deformations around the Einstein--Hilbert core rather than as an entirely different base geometry.
\end{itemize}

\section{Mechanized Artifacts (Cubical Agda + Executable Checkers)}

\subsection*{Artifact A: \texttt{Logic/ModalCollapse.agda}}
This artifact is mechanized in Cubical Agda. It defines explicit Flow/Shape modal predicates over tangent data and proves Euclidean modal collapse via a univalence bridge while separately proving Lorentzian no-collapse under null-cone separation assumptions.

\subsection*{Artifact B: \texttt{Geometry/HopfCeiling.agda}}
This artifact is mechanized in Cubical Agda. It encodes principal-bundle-style cocycle checking with a strictly associative operation interface and proves a 4-fold consistency lemma. A non-associative interface is mechanized separately and requires an explicit coherence payload to derive the analogous result.

\subsection*{Artifact C: \texttt{Haskell/HodgeEndomorphism.hs}}
This executable Haskell artifact enumerates form-degree signatures by dimension and checks the Hodge map on 2-forms. The verifier reports that only $d=4$ yields a native $\Omega^2\to\Omega^2$ endomorphism and therefore supports self/anti-self-dual projectors without extra constructors.

\subsection*{Artifact D: \texttt{Haskell/MinimalAction.hs}}
This executable Haskell artifact generates curvature-scalar candidates, computes AST node complexity, and filters by explicit constraints (local scalar invariant, presence of a curvature term, second-order equations, divergence compatibility). Under these constraints it certifies Einstein--Hilbert minimality.

\begin{center}
\fbox{\parbox{0.92\textwidth}{
\textbf{Mechanization status.} Artifacts A and B are mechanized and typechecked in Cubical Agda; Artifacts C and D are implemented as executable Haskell verifiers used to certify their stated computational claims. We do not claim formalization of the full exotic-\(\R^4\) or full classical Lovelock theorem libraries at present.
}}
\end{center}

\section{Conclusion}

With corrected geometry, consistent indexing, and realistic formalization scope, the PEN derivation becomes sharper: linear quantum kinematics maximizes resource-sensitive novelty, four dimensions maximize reusable structure plus Hodge gain, Lorentzian signature preserves modal logic, and Einstein--Hilbert dynamics minimizes syntactic cost. The architecture of spacetime is therefore treated as a constrained optimum, not an arbitrary backdrop.

\begin{thebibliography}{99}
% BibTeX-ready equivalents for these keys are provided in gr_and_qm_refs.bib.
\bibitem{pen-unified}
H.S.~Lande, ``The Principle of Efficient Novelty: The Algorithmic Origin of the Kinematic Framework of Physics,'' \texttt{pen\_unified.tex}, 2026.

\bibitem{girard-linear-logic}
J.-Y.~Girard, ``Linear Logic,'' \emph{Theoretical Computer Science} \textbf{50} (1987), 1--101.

\bibitem{wootters-zurek}
W.K.~Wootters and W.H.~Zurek, ``A Single Quantum Cannot be Cloned,'' \emph{Nature} \textbf{299} (1982), 802--803.

\bibitem{dieks}
D.~Dieks, ``Communication by EPR Devices,'' \emph{Physics Letters A} \textbf{92} (1982), 271--272.

\bibitem{hottbook}
The Univalent Foundations Program, ``Homotopy Type Theory: Univalent Foundations of Mathematics,'' Institute for Advanced Study, 2013.

\bibitem{atiyah-hitchin-singer}
M.F.~Atiyah, N.J.~Hitchin, and I.M.~Singer, ``Self-Duality in Four-Dimensional Riemannian Geometry,'' \emph{Proceedings of the Royal Society A} \textbf{362} (1978), 425--461.

\bibitem{donaldson-kronheimer}
S.K.~Donaldson and P.B.~Kronheimer, \emph{The Geometry of Four-Manifolds}, Oxford University Press, 1990.

\bibitem{hawking-ellis}
S.W.~Hawking and G.F.R.~Ellis, \emph{The Large Scale Structure of Space-Time}, Cambridge University Press, 1973.

\bibitem{wald-gr}
R.M.~Wald, \emph{General Relativity}, University of Chicago Press, 1984.

\bibitem{lovelock-1971}
D.~Lovelock, ``The Einstein Tensor and Its Generalizations,'' \emph{Journal of Mathematical Physics} \textbf{12} (1971), 498--501.
\end{thebibliography}

\end{document}
