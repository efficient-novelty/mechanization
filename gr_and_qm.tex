\documentclass[12pt,a4paper]{article}

\usepackage[utf8]{inputenc}
\usepackage[T1]{fontenc}
\usepackage{amsmath,amssymb,amsthm}
\usepackage{mathtools}
\usepackage{geometry}
\usepackage{hyperref}
\usepackage{enumitem}
\usepackage{xcolor}
\usepackage{fancyhdr}
\usepackage{titlesec}
\usepackage{abstract}

\geometry{left=2.5cm,right=2.5cm,top=2.5cm,bottom=2.5cm}
\hypersetup{colorlinks=true,linkcolor=blue!70!black,citecolor=green!50!black,urlcolor=blue!70!black}

\newtheorem{theorem}{Theorem}[section]
\newtheorem{proposition}[theorem]{Proposition}
\theoremstyle{definition}
\newtheorem{definition}[theorem]{Definition}

\newcommand{\R}{\mathbb{R}}
\newcommand{\C}{\mathbb{C}}
\newcommand{\HH}{\mathbb{H}}
\newcommand{\OO}{\mathbb{O}}
\newcommand{\flatmod}{\flat}
\newcommand{\nextmod}{\bigcirc}

\titleformat{\section}{\large\bfseries}{\thesection.}{0.5em}{}
\titleformat{\subsection}{\normalsize\bfseries}{\thesubsection}{0.5em}{}

\pagestyle{fancy}
\fancyhf{}
\fancyhead[L]{\small Structural Emergence of 4D Lorentzian Geometry}
\fancyhead[R]{\small H.S.~Lande}
\fancyfoot[C]{\thepage}
\renewcommand{\headrulewidth}{0.4pt}

\begin{document}

\begin{center}
    {\LARGE\bfseries Structural Emergence of 4D Lorentzian Geometry\\[0.3em] from the Principle of Efficient Novelty}

    \vspace{1.5em}

    {\large Halvor S.~Lande}\\[0.5em]
    {\normalsize \href{mailto:hsl@awc.no}{hsl@awc.no}}

    \vspace{1em}

    {\normalsize March 1, 2026}
\end{center}

\vspace{1.5em}

\begin{abstract}
\noindent
We derive quantum kinematics, four-dimensionality, Lorentzian signature, and Einstein dynamics from the Principle of Efficient Novelty (PEN) using the spectral efficiency ratio $\rho=\nu/\kappa$ with $\nu=\nu_G+\nu_H+\nu_C$. The argument is presented in a constructive mechanization-oriented form with explicit artifact support and consistent Step~13/14/15 indexing across companion PEN materials. The resulting claim is: (i) linear/quantum kinematics is selected because Cartesian cloning collapses higher homotopical novelty ($\nu_H\to 0$); (ii) $d=4$ is selected by a native-associativity cost ceiling plus a 4D Hodge endomorphism spike in $\nu_C$; (iii) Lorentzian signature is required to prevent univalent modal collapse of Flow and Cohesion; and (iv) Einstein--Hilbert dynamics is selected as the MBTT-shortest curvature action.
\end{abstract}

\vspace{1em}
\hrule
\vspace{2em}

\section{Introduction}

\subsection{Scope: Kinematic Library vs Parametric Instantiation}

The foundational text, \emph{The Principle of Efficient Novelty: The Algorithmic Origin of the Kinematic Framework of Physics}, is explicit (Remark~2.1) that PEN does not directly derive fixed physical constants such as a unique spacetime dimension or equations of motion at the library-construction stage. This is not a contradiction with the present paper: the foundational work derives a \emph{Kinematic Library} (generic reusable operators, metrics, cohesion, Hilbert functionals, and DCT), while this companion evaluates a \emph{Parametric Instantiation} of that library under the same selection rule $\max\rho$.

In that second-stage instantiation problem, candidate architectures are scored by
\begin{equation}
\rho = \frac{\nu_G+\nu_H+\nu_C}{\kappa},
\end{equation}
where $\nu_G$ counts new Introduction schemas, $\nu_H$ new homotopy/computation schemas, $\nu_C$ new Elimination schemas, and $\kappa$ MBTT specification complexity. Under this unchanged PEN mechanics, the physical optimum isolates $d=4$, Lorentzian signature, and Einstein--Hilbert dynamics.

\subsection{Index Alignment with the Companion Paper}

We use the foundational indexing directly: \textbf{Step 13} (Metric/Frame), \textbf{Step 14} (Hilbert functional), and \textbf{Step 15} (Dynamical Cohesive Topos). This standardization is used throughout.

\section{Quantum Kinematics from Efficient Novelty}

\subsection{No-Cloning as a Type-Theoretic Constraint}

In homotopy type theory, imposing strict global cloning coherence drives types toward h-sets (0-groupoids), suppressing higher path structure \cite{hottbook}. In PEN spectral language this is exactly the collapse channel identified by Proposition~9.2 (\emph{Inefficiency of Discrete Structures}): forcing discreteness sends topological novelty to
\begin{equation}
\nu_H=0,
\end{equation}
and the branch cannot compete once the efficiency threshold rises.

The operational no-cloning pressure in quantum theory is classically captured by standard no-cloning results \cite{wootters-zurek,dieks}, while resource-sensitive proof theory is rooted in linear logic \cite{girard-linear-logic}. PEN's point is algorithmic: no-cloning preserves higher-path channels, so $\nu_H$ can remain superlinear in representational depth instead of collapsing to zero.

\subsection{Linearization and the $d=2$ Coherence Window}

By Theorem~4.2 (the $d=2$ Coherence Window in the foundational paper), native foundations cannot resolve 3-dimensional coherence obligations without explicit overhead. In discrete/h-set collapse regimes, integration debt follows the Fibonacci barrier $\Delta_n=F_n$, so purely discrete encodings cannot outpace this cost. Linear/quantum kinematics is therefore a survival condition: avoid cloning-induced modal flattening to preserve the homotopy amplification channel ($\nu_H\sim\mathcal{O}(d^2)$ in the non-collapsed regime) while paying only bounded incremental $\kappa$.

\begin{center}
\fbox{\parbox{0.92\textwidth}{
\textbf{Section conclusion.} Quantum kinematics is selected because it prevents cloning-induced $\nu_H\!=\!0$ collapse and preserves high-yield homotopy amplification relative to coherence cost.
}}
\end{center}

\section{Dimensionality: Why Four}

\subsection{The Native Associativity Ceiling and the Quaternionic Base}

Pure differential geometry does \emph{not} cap dimension by division algebras: $GL(n,\R)$ exists for all $n$ and is associative. PEN's selection statement is about primitive synthesis cost: if high-$n$ patching is built from scratch, $\kappa$ rises sharply.

Theorem~4.2 (Coherence Window $d=2$) makes this precise for non-associative patching. Octonionic patching requires explicit higher coherence towers (Stasheff $A_\infty$ data) because associators cannot be discharged natively in a 2-dimensional coherence window. That explicit coherence payload is AST-heavy and induces a large $\kappa$ penalty. By contrast, quaternionic $S^3$-controlled fibrations are associative and remain within native coherence handling.

Hence the native associativity ceiling is reached at quaternionic control, geometrically tied to
\begin{equation}
S^3 \hookrightarrow S^7 \to S^4,
\end{equation}
with $S^4$ as the maximal high-yield base dimension before non-native coherence costs dominate.

\subsection{Algebraic Optimality of the 4D Hodge Endomorphism}

The tractable and sufficient 4D gain is algebraic:
\begin{equation}
\star : \Omega^2(M^d) \to \Omega^{d-2}(M^d).
\end{equation}
Only at $d=4$ does $\star$ close on 2-forms,
\begin{equation}
\star : \Omega^2 \to \Omega^2,
\end{equation}
producing self-dual/anti-self-dual projectors \cite{atiyah-hitchin-singer,donaldson-kronheimer}. In PEN accounting this is a sharp $\nu_C$ spike: Step~13 already pays the metric/Hodge infrastructure cost, so the 4D closure unlocks many new elimination channels at marginal $\Delta\kappa\approx 0$ (instanton sector decompositions, chirality-sensitive gauge decompositions, and projector-based curvature factorizations).

Artifact~C (\texttt{Haskell/HodgeEndomorphism.hs}) computes this exactly and reports that among tested dimensions only $d=4$ yields native $\Omega^2\to\Omega^2$ closure, confirming the concentrated $\nu_C$ gain at fixed constructor budget.

\begin{center}
\fbox{\parbox{0.92\textwidth}{
\textbf{Section conclusion.} Dimension four is selected at the intersection of a strict $\kappa$ bound (associativity/coherence ceiling) and a unique 4D $\nu_C$ bonus from Hodge self-duality.
}}
\end{center}

\section{Signature: Lorentzian as Modal Separation Geometry}

\subsection{Step 15 Modal Structure, Tangent Time, and Univalence}

Step~15 (DCT) carries distinct modalities: Cohesion ($\flatmod$) for extended shape and Flow ($\nextmod$) for temporal update. The Kinematic-Dynamic Equivalence Lemma (Lemma~5.10 in the foundational text) identifies time-internalization as
\begin{equation}
\nextmod X \simeq X^{\mathbb{D}},
\end{equation}
so Flow is geometrically bound to tangent data. A physical metric must preserve this modal distinction rather than quotienting it away.

By univalence, coherent geometric paths induce type equivalences \cite{hottbook}. In Euclidean signature $(++++)$, $SO(4)$ isotropy continuously rotates a timelike-designated tangent vector into a spacelike direction. Via Lemma~5.10, that smooth geometric rotation lifts to a universe path in $\mathcal{U}$ identifying the Flow generator with a spatial/cohesive direction, driving modal collapse $\nextmod\simeq\flatmod$.

\subsection{Light Cone as a Type-Theoretic Firewall}

Lorentzian signature $(-+++)$ partitions tangent vectors into timelike/spacelike/null classes. Under $SO(1,3)$, timelike and spacelike components are separated by the null cone; no causal-class-preserving path crosses this barrier \cite{hawking-ellis,wald-gr}. The univalent identification path needed for $\nextmod\simeq\flatmod$ is therefore topologically absent, and Step~15 modal structure remains protected.

Artifact~A (\texttt{Logic/ModalCollapse.agda}) implements both branches: Euclidean collapse is derivable under rotational isotropy, whereas Lorentzian no-collapse is verified under null-cone separation hypotheses.

\begin{center}
\fbox{\parbox{0.92\textwidth}{
\textbf{Section conclusion.} Lorentzian anisotropy is selected as the minimal geometric firewall that preserves Step~15 modal separation.
}}
\end{center}

\section{Dynamics: Einstein--Hilbert as MBTT-AST Minimum}

Rather than relying on non-constructive classical Lovelock uniqueness theorems, PEN evaluates a computable AST minimality statement. Under the prefix-free Minimal Binary Type Theory (MBTT) grammar from the foundational paper, among local scalar curvature actions yielding second-order divergence-compatible 4D dynamics, the Einstein--Hilbert term
\begin{equation}
S_{\mathrm{EH}} = \int (R - 2\Lambda)\,\sqrt{-g}\,d^4x
\end{equation}
has strictly shortest encoding and lowest constructor depth.

Higher-curvature terms such as $R^2$ or $R_{\mu\nu}R^{\mu\nu}$ add AST nesting and increase $\kappa$ without introducing comparably new elimination families $\nu_C$ for the second-order sector, so $\rho$ drops. Artifact~D (\texttt{Haskell/MinimalAction.hs}) computes candidate AST node counts and verifies this strict minimality under the explicit filter set (local scalar invariant, curvature dependence, second-order equations, divergence compatibility), in compatibility with Lovelock context \cite{lovelock-1971}.

\section{Predictions}

\begin{itemize}[leftmargin=2em]
    \item \textbf{No accessible extra dimensions:} higher-dimensional proposals cross the associativity/coherence $\kappa$ ceiling without compensating spectral gain.
    \item \textbf{No Lorentz-violation regime:} breaking Lorentzian structure reopens univalent modal-collapse channels.
    \item \textbf{Minimal gravity sector first:} leading UV-consistent corrections, if any, should appear as controlled high-$\kappa$ deformations around the Einstein--Hilbert core.
\end{itemize}

\section{Mechanized Artifacts (Cubical Agda + Executable Checkers)}

\subsection*{Artifact A: \texttt{Logic/ModalCollapse.agda}}
Mechanized in Cubical Agda. Defines explicit Flow/Shape modal predicates over tangent data and proves Euclidean collapse via a univalence bridge while separately proving Lorentzian no-collapse under null-cone separation assumptions.

\subsection*{Artifact B: \texttt{Geometry/HopfCeiling.agda}}
Mechanized in Cubical Agda. Encodes principal-bundle cocycle checking with a strictly associative operation interface and proves a 4-fold consistency lemma. A non-associative interface is mechanized separately and requires explicit coherence payload to derive analogous consistency.

\subsection*{Artifact C: \texttt{Haskell/HodgeEndomorphism.hs}}
Executable Haskell checker. Enumerates form-degree signatures by dimension and verifies that only $d=4$ yields native $\Omega^2\to\Omega^2$ closure.

\subsection*{Artifact D: \texttt{Haskell/MinimalAction.hs}}
Executable Haskell checker. Generates curvature-scalar candidates, computes AST node complexity, and filters by explicit constraints, certifying Einstein--Hilbert minimality in the admissible set.

\begin{center}
\fbox{\parbox{0.92\textwidth}{
\textbf{Mechanization status.} Artifacts A and B are mechanized and typechecked in Cubical Agda; Artifacts C and D are executable Haskell verifiers used for quantitative certification of the 4D Hodge closure and MBTT-AST minimality claims.
}}
\end{center}

\section{Conclusion}

PEN's foundational paper derives the reusable kinematic language; this companion solves the physical instantiation optimization in that language. Under the same spectral metrics $(\nu_G,\nu_H,\nu_C,\kappa)$ and selection rule $\max\rho$, quantum kinematics survives by preventing $\nu_H$ collapse, $d=4$ is pinned by associativity/coherence bounds plus a unique $\nu_C$ Hodge spike, Lorentzian signature protects Step~15 modal logic, and Einstein--Hilbert is MBTT-minimal among admissible second-order curvature laws. Spacetime architecture is thus a constrained efficiency optimum, not an arbitrary backdrop.

\begin{thebibliography}{99}
% BibTeX-ready equivalents for these keys are provided in gr_and_qm_refs.bib.
\bibitem{pen-unified}
H.S.~Lande, ``The Principle of Efficient Novelty: The Algorithmic Origin of the Kinematic Framework of Physics,'' \texttt{pen\_unified.tex}, 2026.

\bibitem{girard-linear-logic}
J.-Y.~Girard, ``Linear Logic,'' \emph{Theoretical Computer Science} \textbf{50} (1987), 1--101.

\bibitem{wootters-zurek}
W.K.~Wootters and W.H.~Zurek, ``A Single Quantum Cannot be Cloned,'' \emph{Nature} \textbf{299} (1982), 802--803.

\bibitem{dieks}
D.~Dieks, ``Communication by EPR Devices,'' \emph{Physics Letters A} \textbf{92} (1982), 271--272.

\bibitem{hottbook}
The Univalent Foundations Program, ``Homotopy Type Theory: Univalent Foundations of Mathematics,'' Institute for Advanced Study, 2013.

\bibitem{atiyah-hitchin-singer}
M.F.~Atiyah, N.J.~Hitchin, and I.M.~Singer, ``Self-Duality in Four-Dimensional Riemannian Geometry,'' \emph{Proceedings of the Royal Society A} \textbf{362} (1978), 425--461.

\bibitem{donaldson-kronheimer}
S.K.~Donaldson and P.B.~Kronheimer, \emph{The Geometry of Four-Manifolds}, Oxford University Press, 1990.

\bibitem{hawking-ellis}
S.W.~Hawking and G.F.R.~Ellis, \emph{The Large Scale Structure of Space-Time}, Cambridge University Press, 1973.

\bibitem{wald-gr}
R.M.~Wald, \emph{General Relativity}, University of Chicago Press, 1984.

\bibitem{lovelock-1971}
D.~Lovelock, ``The Einstein Tensor and Its Generalizations,'' \emph{Journal of Mathematical Physics} \textbf{12} (1971), 498--501.
\end{thebibliography}

\end{document}
